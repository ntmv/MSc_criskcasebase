\documentclass[AMA,Times1COL]{WileyNJDv5} %STIX1COL,STIX2COL,STIXSMALL


\articletype{Article Type}%

\received{Date Month Year}
\revised{Date Month Year}
\accepted{Date Month Year}
\journal{Journal}
\volume{00}
\copyyear{2023}
\startpage{1}

\raggedbottom



\begin{document}

\title{Penalized Competing Risks Analysis using  casebase sampling}

\author[1]{Author One}

\author[2,3]{Author Two}

\author[3]{Author Three}

\authormark{TAYLOR \textsc{et al.}}
\titlemark{PLEASE INSERT YOUR ARTICLE TITLE HERE}

\address[1]{\orgdiv{Department Name}, \orgname{Institution Name}, \orgaddress{\state{State Name}, \country{Country Name}}}

\address[2]{\orgdiv{Department Name}, \orgname{Institution Name}, \orgaddress{\state{State Name}, \country{Country Name}}}

\address[3]{\orgdiv{Department Name}, \orgname{Institution Name}, \orgaddress{\state{State Name}, \country{Country Name}}}

\corres{Corresponding author Mark Taylor, This is sample corresponding address. \email{authorone@gmail.com}}

\presentaddress{This is sample for present address text this is sample for present address text.}

%\fundingInfo{Text}
%\JELinfo{ejlje}

\abstract[Abstract]{In biomedical studies, quantifying the association between prognostic genes and markers with time-to-event outcomes is crucial for predicting a patient's disease risk based on their specific covariate profile. Modeling competing risks is necessary, as patients may face multiple mutually exclusive events, such as death from different causes. However, current methods for competing risks analysis often yield coefficient estimates that are difficult to interpret, making it challenging to connect them to event rates. Additionally, the high dimensionality of genomic data, where the number of variables exceeds the number of subjects, presents a significant challenge.

In this work, we propose a novel approach using an elastic-net penalized multinomial model within a casebased sampling framework to analyze competing risks survival data. We also introduce a two-step method called the de-biased case-base to improve the predictive performance regarding disease risk. Through a comprehensive simulation study that replicates biomedical data, we show that the casebase method is effective in variable selection and survival prediction, particularly in scenarios involving non-proportional hazards. Moreover, we highlight the flexibility of this approach in generating smooth-time incidence curves, which significantly improve the accuracy of patient risk estimation assessments.}

\keywords{keyword1, keyword2, keyword3, keyword4}

\jnlcitation{\cname{%
\author{Taylor M.},
\author{Lauritzen P},
\author{Erath C}, and
\author{Mittal R}}.
\ctitle{On simplifying ‘incremental remap’-based transport schemes.} \cjournal{\it J Comput Phys.} \cvol{2021;00(00):1--18}.}


\maketitle

\renewcommand\thefootnote{}
\footnotetext{\textbf{Abbreviations:} ANA, anti-nuclear antibodies; APC, antigen-presenting cells; IRF, interferon regulatory factor.}

\renewcommand\thefootnote{\fnsymbol{footnote}}
\setcounter{footnote}{1}

\section{INTRODUCTION}\label{sec1}

In numerous clinical studies, a primary outcome frequently modeled is the time to an event of interest, or survival time during the study period. When every individual in a study experiences the event, various statistical methods can be applied effectively. However, analyzing survival data presents certain challenges: 1) Not all patients experience the event within the study timeframe, yet they remain at risk of doing so in the future, leaving their actual time to the event unknown. 2) Survival data are rarely normally distributed and often exhibit a skewed distribution.

Given these challenges, developing methods specifically tailored for survival data is essential. In typical survival or time-to-event analysis, the outcome variable indicates the time until one event of interest occurs. However, combined endpoints are a common extension that includes both the cause of interest and a competing risk. In this context, a competing risk is an event that prevents the primary event of interest from happening. For example, many medical studies consider 'disease-free survival,' defined as the time until either the recurrence of a disease or death (from a cause other than the disease), whichever occurs first. Here, while death due to the disease is our primary outcome, death from another cause is classified as a competing risk.

From a clinical perspective, the most important quantity for both clinicians and patients derived from survival analysis is the estimated risk of an event for patients with a particular covariate profile. This risk can be estimated through the cumulative incidence function (CIF), also known as absolute risk, which indicates the probability that a person at risk will experience the event of interest within a specified time frame.

One widely used method for analyzing a single endpoint in survival analysis is the Cox Proportional Hazards model. Its flexibility comes from its semi-parametric formulation, enabling inference about how covariates influence the hazard function without restricting the response to a specific parametric family. However, this flexibility has a drawback: the baseline hazard is separated from the covariate effects, requiring separate estimation of the baseline hazard when calculating the cumulative incidence function. This results in stepwise survival risk estimates that can be difficult to interpret. Additionally, extending the Cox model to a competing risks scenario involves fitting separate models for each risk. This approach is limited because, although we can estimate cause-specific hazards, we cannot simultaneously estimate the cumulative incidence of the event of interest while accounting for competing risks.

To overcome these limitations, the Fine-Gray model was introduced as an alternative, directly modelling the impact of covariates on the cumulative incidence function. However, when assessing whether a particular risk factor is associated with the rate of outcome occurrence among event-free individuals, cause-specific hazards remain relevant, which this model does not readily provide. Furthermore, certain combinations of covariates in the Fine-Gray model can produce cumulative incidence probability estimates exceeding 1. In fact, in modelling competing risks, existing approaches typically focus on cause-specific hazards or cumulative incidence functions. Since comprehensive competing risks survival analysis requires both, estimating one from the other with current models is not straightforward.

As survival data, particularly high-dimensional data, become more prevalent, there is a growing need for a competing risks model that can effectively handle cause-specific hazards, produce smooth and interpretable estimates of the cumulative incidence function, and be adaptable for high-dimensional data settings and scenarios involving time-varying covariates. Therefore, we propose using a multinomial model combined with casebase sampling to fit fully parametric hazard models. This approach provides smooth hazard and cumulative incidence function estimates, and our application of multinomial regression enables access to the desirable properties of the generalized linear model family, including regularisation-based variable selection.

This paper shows that the multinomial model with casebase sampling exhibits strong variable selection in both low- and high-dimensional data, while effectively predicting cumulative risk, making it a viable option for comprehensive competing risks analysis. The structure of this paper includes an explanation of the casebase framework and its multinomial implementation, along with the debiasing step. It then presents the results of our simulation experiments, focusing on variable selection and prediction performance, and explores the model's ability to estimate the cumulative incidence function by debiasing the casebase model. Subsequently, this model is applied to the xxxx dataset to evaluate its performance in real-world medical data. Finally, the paper discusses the limitations, potential extensions of the approach, and suggestions for future research. 


\section{METHODS}\label{sec2}

\subsection{Casebase Framework}

In a competing risks setting, we model the occurrence of different event types using counting processes. Let $N_{0j;i}(t)$ be the counting process for individual $i$ and event type $j$. This process is equal to 1 if individual $i$ transitions from a starting state (state 0) to event state $j$ within the time interval $[0, t]$, and 0 otherwise. The history of the process for individual $i$ up to time $t$ is denoted by $\mathcal{H}_{i}(t-)$. The expected differential change in the counting process relates to the cause-specific hazard function, $\alpha_{ij}(t)$, through the following expression
$$E[dN_{0j;i}(t) | \mathcal{H}_{i}(t-)] = I(T_i \ge t, J_i = j) \alpha_{ij}(t)dt.$$

The casebase sampling framework was formally proposed for fitting smooth-in-time functions. It extends methods used in case-control studies. This framework considers an entire study base, which is the total of all individual follow-up times, or person-time. The study base consists of person-moments. A person-moment is defined as a specific point in time, $t$, and the individual's corresponding prognostic value at that time.

From the study base, two series are generated: a case series and a base series. The case series includes all person-moments where an event of interest occurred. The base series is made up of a representative sample of person-moments, called controls, taken from the study base. These person-moments are sampled by first selecting an individual $i$ with probability $\phi_{i}=B_{i}/B$, where $B_{i}$ is the total person-time contributed by individual $i$, and $B$ is the total person-time in the study. A person-moment $b_j$ is then sampled uniformly from that individual's follow-up time, a process that can be modelled as $b_{j}\sim \text{Unif}(0,B_i)$. An individual might contribute multiple person-moments to the base series.

The fundamental idea of casebase sampling is comparing person-moments when events occurred (the cases) with person-moments when individuals were at risk of the event (the bases). This method of representative sampling differs from the risk-set sampling used in Cox regression and aims to capture the entire baseline hazard. A key benefit of this approach is its capacity to directly estimate cause-specific hazard functions, which then enables the derivation of smooth-in-time survival and cumulative incidence functions.

To estimate the parameters, we revisit the counting process framework. The cause-specific hazard function for event $j$ and individual $i$ is parameterized using a vector of coefficients, $\beta_{j}$, and time, $t$, denoted as $\alpha_{ij}(t;\beta_{j})$. According to the relationship established for counting processes, the differential change in the cause-specific hazard functions for individual $i$ at time-to-event $T_i$ satisfies
$$\alpha_{ij}(t;\beta_{j})dt=E[dN_{0j;i}(t)|\mathcal{H}_{i}(t-)]$$
This provides the foundation for building the likelihood and estimating functions for the model parameters.

\subsection{Multinomial Parameterization, Regularization and Optimization}

We can construct a multinomial logistic model for the casebase likelihood. We use $Y_{i}$ to denote a categorical response variable. Let us define $Y_{i}$ to have three levels for the case of two competing events, i.e., {0,1,2}, where the total number of competing events is two. The likelihood function can be written as:
$$\log\left(\frac{\Pr(Y=j|X_i)}{\Pr(Y=0|X_i)}\right)$$ where class 0, i.e., the censored individuals, serves as the reference class. The likelihood is defined for a set of covariates $X_{i}$ for individual $i$.

In terms of optimizing a penalized likelihood, the glmnet package has implemented fast algorithms for several generalized linear models, including multinomial regression with the elastic-net family penalty. However, the glmnet package uses a symmetric parameterization of the multinomial model. This parameterization estimates the relative differences between classes rather than the absolute probabilities for each class, resulting in the constant offset term not being fitted. Since the case base approach relies on the constant offset of the intensity function, we need a penalized model using the multinomial logistic parameterization. Developing a function to fit and tune this penalized model is a key contribution of this work.

The penalized multinomial logistic regression is fitted using accelerated stochastic variance reduced gradient descent (ASVRG), as this algorithm shows fast convergence for high-dimensional datasets with a larger number of predictors than observations ($p> n$). This accelerated method combines SVRG's variance reduction with momentum to speed up convergence (@DriggsEhrhardtSchonlieb:2022). Like SVRG, this algorithm begins with a full-batch gradient computation. However, instead of relying solely on stochastic updates, it employs a sequence of extrapolated points, constructed as a weighted combination of the current and previous iterations.

Here, we denote $l(\beta)$ as the joint likelihood across all individuals $i=1,...,n$ and causes $j=1,2$, i.e., $\sum_{i=1}^{n}\sum}_{j=1}^{2}\log(\frac{\text{Pr}(Y_{i}=j|X_{i})}{\text{Pr}(Y_{i}=0|X_{i})})$. For covariates $k\in\{1,...,p\}$, we can estimate the coefficient matrix $eta$ as
$$\hat{\beta}=\text{arg max}_{\beta}\{l(\beta)+\lambda\sum_{k=1}^{p}w_{k}((\frac{1-\phi}{2})\sum_{j=1}^{2}\beta_{kj}^{2}+\phi\sum_{j=1}^{2}|\beta_{kj}|)\}.$$

$\phi$ is the mixing parameter between the LASSO and ridge penalties. Setting $\phi=1$ and $\phi=0$ corresponds to the LASSO and ridge penalties, respectively. $w_{k}$ represents the penalty factor for the $k^{th}$ covariate, allowing parameters to be penalized differently. In this work, the penalty factor for the intercept and for time is set to 0, so they are unpenalized and always included in the model. Solving for the coefficients $eta$ will produce a matrix of size $(p+1)\times2$, where the $j^{th}$ column corresponds to the coefficients for cause j, for $j=1,2$.

A cross-validation function was also implemented to fine-tune the shrinkage parameter $\lambda$. The tuning relies on the multinomial deviance as the measure of model goodness-of-fit, shown below:
$$-2\sum_{i=1}^{n}\sum_{j=1}^{2}\log(\frac{\text{Pr}(Y_{i}=j|X_{i})}{\text{Pr}(Y_{i}=0|X_{i})}).$$

\subsection{De-biasing Step}

Pellentesque habitant morbi tristique senectus et netus et malesuada fames ac turpis egestas. Donec odio elit, dictum
in, hendrerit sit amet, egestas sed, leo. Praesent feugiat sapien aliquet odio. Integer vitae justo. Aliquam vestibulum
fringilla lorem. Sed neque lectus, consectetuer at, consectetuer sed, eleifend ac, lectus. Nulla facilisi. Pellentesque
eget lectus. Proin eu metus. Sed porttitor. In hac habitasse platea dictumst. Suspendisse eu lectus. Ut mi mi, lacinia
sit amet, placerat et, mollis vitae, dui. Sed ante tellus, tristique ut, iaculis eu, malesuada ac, dui. Mauris nibh leo,
facilisis non, adipiscing quis, ultrices a, dui.


\section{SIMULATION STUDIES}\label{sec3}

\subsection{kjlj}

Data are simulated from a $K=2$ competing risks proportional hazards model. The cause-specific hazard for cause for individual $i$ with covariates $X_i = (X_{i1}, \dots, X_{ip})$ at time $t$ is $\lambda_k(t | X_i) = \lambda_{0k}(t) \exp(X_i^T \beta_k)$. The baseline hazards $\lambda_{0k}(t)$ follow a Weibull distribution $\lambda_{0k}(t) = h_k \gamma_k t^{\gamma_k - 1}$, with parameters $(h_1, \gamma_1)=(0.55, 1.5)$ and $(h_2, \gamma_2)=(0.35, 1.5)$.

The coefficient vectors $\beta_1, \beta_2 \in \mathbb{R}^p$ are sparse, with non-zero effects predominantly within the first 18 predictors. For cause 1, the coefficients for $X_1, \dots, X_{18}$ are set as $(1, 1, 1, 1, 1, 1,$ $0.5, -0.5, 0.5, -0.5, 0.5, -0.5$ $1, 1, 1, 1, 1, 1)$, and 0 for $j > 18$. For cause 2 ($\beta_2$), the coefficients for $X_1, \dots, X_{24}$ are set as $(0, 0, 0, 0, 0, 0,$ $0.5, -0.5, 0.5, -0.5, 0.5,$ $-0.5, 0, 0, 0, 0, 0, 0,$ $1, 1, 1, 1, 1, 1)$, and 0 for $j > 24$.

Event times $T_i$ and causes $C_i$ are generated by simulating potential failure times $T_{ik}$ from $\lambda_k(t|X_i)$ and setting $T_i = \min(T_{i1}, T_{i2})$ with $C_i$ being the index $k$ for which $T_{ik} = T_i$. Independent censoring times $T_{cens, i}$ are generated based on an overall rate of 0.05. Observed data consist of $(ftime_i, fstatus_i)$, where $ftime_i = \min(T_i, T_{cens, i})$ and the status $fstatus_i = C_i \cdot \mathds{1}(T_i \le T_{cens, i})$ (with $fstatus_i=0$ indicating censoring).


\subsection{Second level head}

Etiam euismod. Fusce facilisis lacinia dui. Suspendisse potenti. In mi erat, cursus id, nonummy sed, ullamcorper
eget, sapien. Praesent pretium, magna in eleifend egestas, pede pede pretium lorem, quis consectetuer tortor sapien
facilisis magna. Mauris quis magna varius nulla scelerisque imperdiet. Aliquam non quam. Aliquam porttitor quam
a lacus. Praesent vel arcu ut tortor cursus volutpat. In vitae pede quis diam bibendum placerat. Fusce elementum
convallis neque. Sed dolor orci, scelerisque ac, dapibus nec, ultricies ut, mi. Duis nec dui quis leo sagittis commodo.
Nulla non mauris vitae wisi posuere convallis. Sed eu nulla nec eros scelerisque pharetra. Nullam varius. Etiam
dignissim elementum metus. Vestibulum faucibus, metus sit amet mattis rhoncus, sapien dui laoreet odio, nec ultricies
nibh augue a enim. Fusce in ligula. Quisque at magna et nulla commodo consequat. Proin accumsan imperdiet sem.
Nunc porta. Donec feugiat mi at justo. Phasellus facilisis ipsum quis ante. In ac elit eget ipsum pharetra faucibus.
Maecenas viverra nulla in massa (Table~\ref{tab2}).

\begin{definition}
Example definition text. Example definition text. Example definition text. Example definition text. Example definition text. Example definition text. Example definition text. Example definition text. Example definition text. Example definition text. Example definition text.
\end{definition}

Sed commodo posuere pede. Mauris ut est. Ut quis purus. Sed ac odio. Sed vehicula hendrerit sem. Duis non
odio. Morbi ut dui. Sed accumsan risus eget odio. In hac habitasse platea dictumst. Pellentesque non elit. Fusce
sed justo eu urna porta tincidunt. Mauris felis odio, sollicitudin sed, volutpat a, ornare ac, erat. Morbi quis dolor.
Donec pellentesque, erat ac sagittis semper, nunc dui lobortis purus, quis congue purus metus ultricies tellus. Proin
et quam. Class aptent taciti sociosqu ad litora torquent per conubia nostra, per inceptos hymenaeos. Praesent sapien
turpis, fermentum vel, eleifend faucibus, vehicula eu, lacus.


\begin{proof}
Example for proof text. Example for proof text. Example for proof text. Example for proof text. Example for proof text. Example for proof text. Example for proof text. Example for proof text. Example for proof text. Example for proof text.
\end{proof}

\begin{algorithm}
\caption{\enskip Pseudocode for our algorithm}\label{alg1}
\begin{algorithmic}
  \For each frame
  \For water particles $f_{i}$
  \State compute fluid flow \cite{Hirt1974}
  \State compute fluid--solid interaction \cite{Benson1992}
  \State apply adhesion and surface tension \cite{Margolin2003}
  \EndFor
   \For solid particles $s_{i}$
   \For neighboring water particles $f_{j}$
   \State compute virtual water film \\(see Section~\ref{sec3})
   \EndFor
   \EndFor
   \For solid particles $s_{i}$
   \For neighboring water particles $f_{j}$
   \State compute growth direction vector \\(see Section~\ref{sec2})
   \EndFor
   \EndFor
   \For solid particles $s_{i}$
   \For neighboring water particles $f_{j}$
   \State compute $F_{\theta}$ (see Section~\ref{sec1})
   \State compute $CE(s_{i},f_{j})$ \\(see Section~\ref{sec3})
   \If $CE(b_{i}, f_{j})$ $>$ glaze threshold
   \State $j$th water particle's phase $\Leftarrow$ ICE
   \EndIf
   \If $CE(c_{i}, f_{j})$ $>$ icicle threshold
   \State $j$th water particle's phase $\Leftarrow$ ICE
   \EndIf
   \EndFor
   \EndFor
  \EndFor
\end{algorithmic}
\end{algorithm}



\section{Conclusions}\label{sec5}

Lorem ipsum dolor sit amet, consectetuer adipiscing elit. Ut purus elit, vestibulum ut, placerat ac, adipiscing vitae,
felis. Curabitur dictum gravida mauris. Nam arcu libero, nonummy eget, consectetuer id, vulputate a, magna. Donec
vehicula augue eu neque. Pellentesque habitant morbi tristique senectus et netus et malesuada fames ac turpis egestas.
Mauris ut leo. Cras viverra metus rhoncus sem. Nulla et lectus vestibulum urna fringilla ultrices. Phasellus eu tellus
sit amet tortor gravida placerat. Integer sapien est, iaculis in, pretium quis, viverra ac, nunc. Praesent eget sem vel
leo ultrices bibendum. Aenean faucibus. Morbi dolor nulla, malesuada eu, pulvinar at, mollis ac, nulla. Curabitur
auctor semper nulla. Donec varius orci eget risus. Duis nibh mi, congue eu, accumsan eleifend, sagittis quis, diam.
Duis eget orci sit amet orci dignissim rutrum.


%\backmatter
\bmsection*{Author contributions}

This is an author contribution text. This is an author contribution text. This is an author contribution text. This is an author contribution text. This is an author contribution text.

\bmsection*{Acknowledgments}
This is acknowledgment text. \cite{Kenamond2013} Provide text here. This is acknowledgment text. Provide text here. This is acknowledgment text. Provide text here. This is acknowledgment text. Provide text here. This is acknowledgment text. Provide text here. This is acknowledgment text. Provide text here. This is acknowledgment text. Provide text here. This is acknowledgment text. Provide text here. This is acknowledgment text. Provide text here.


\bmsection*{Financial disclosure}

None reported.

\bmsection*{Conflict of interest}

The authors declare no potential conflict of interests.

\bibliography{wileyNJD-AMA}


\bmsection*{Supporting information}

Additional supporting information may be found in the
online version of the article at the publisher’s website.




\appendix

\bmsection{Program codes appear in Appendix\label{app1}}
\vspace*{12pt}
Using the package {\tt listings} you can add non-formatted text as you would do with \verb|\begin{verbatim}| but its main aim is to include the source code of any programming language within your document.\newline Use \verb|\begin{lstlisting}...\end{lstlisting}| for program codes without mathematics.

The {\tt listings} package supports all the most common languages and it is highly customizable. If you just want to write code within your document, the package provides the {\tt lstlisting} environment; the output will be in Computer Modern typewriter font. Refer to the below example:


\begin{lstlisting}[caption={Descriptive caption text},label=DescriptiveLabel, basicstyle=\fontsize{8}{10}\selectfont\ttfamily]
for i:=maxint to 0 do
begin
{ do nothing }
end;
Write('Case insensitive ');
WritE('Pascal keywords.');
\end{lstlisting}



\bmsubsection{Subsection title of first appendix\label{app1.1a}}

Nam dui ligula, fringilla a, euismod sodales, sollicitudin vel, wisi. Morbi auctor lorem non justo. Nam lacus libero,
pretium at, lobortis vitae, ultricies et, tellus. Donec aliquet, tortor sed accumsan bibendum, erat ligula aliquet magna,
vitae ornare odio metus a mi. Morbi ac orci et nisl hendrerit mollis. Suspendisse ut massa. Cras nec ante. Pellentesque
a nulla. Cum sociis natoque penatibus et magnis dis parturient montes, nascetur ridiculus mus. Aliquam tincidunt
urna. Nulla ullamcorper vestibulum turpis. Pellentesque cursus luctus mauris.

Nulla malesuada porttitor diam. Donec felis erat, congue non, volutpat at, tincidunt tristique, libero. Vivamus
viverra fermentum felis. Donec nonummy pellentesque ante. Phasellus adipiscing semper elit. Proin fermentum massa
ac quam. Sed diam turpis, molestie vitae, placerat a, molestie nec, leo. Maecenas lacinia. Nam ipsum ligula, eleifend
at, accumsan nec, suscipit a, ipsum. Morbi blandit ligula feugiat magna. Nunc eleifend consequat lorem. Sed lacinia
nulla vitae enim. Pellentesque tincidunt purus vel magna. Integer non enim. Praesent euismod nunc eu purus. Donec
bibendum quam in tellus. Nullam cursus pulvinar lectus. Donec et mi. Nam vulputate metus eu enim. Vestibulum
pellentesque felis eu massa.
Nulla malesuada porttitor diam. Donec felis erat, congue non, volutpat at, tincidunt tristique, libero. Vivamus
viverra fermentum felis. Donec nonummy pellentesque ante. Phasellus adipiscing semper elit. Proin fermentum massa
ac quam. Sed diam turpis, molestie vitae, placerat a, molestie nec, leo. Maecenas lacinia. Nam ipsum ligula, eleifend
at, accumsan nec, suscipit a, ipsum. Morbi blandit ligula feugiat magna. Nunc eleifend consequat lorem. Sed lacinia
nulla vitae enim. Pellentesque tincidunt purus vel magna. Integer non enim. Praesent euismod nunc eu purus. Donec
bibendum quam in tellus. Nullam cursus pulvinar lectus. Donec et mi. Nam vulputate metus eu enim. Vestibulum
pellentesque felis eu massa.

\bmsubsubsection{Subsection title of first appendix\label{app1.1.1a}}

\noindent\textbf{Unnumbered figure}


\begin{center}
\includegraphics[width=7pc,height=8pc,draft]{empty}
\end{center}


Fusce mauris. Vestibulum luctus nibh at lectus. Sed bibendum, nulla a faucibus semper, leo velit ultricies tellus, ac
venenatis arcu wisi vel nisl. Vestibulum diam. Aliquam pellentesque, augue quis sagittis posuere, turpis lacus congue
quam, in hendrerit risus eros eget felis. Maecenas eget erat in sapien mattis porttitor. Vestibulum porttitor. Nulla
facilisi. Sed a turpis eu lacus commodo facilisis. Morbi fringilla, wisi in dignissim interdum, justo lectus sagittis dui, et
vehicula libero dui cursus dui. Mauris tempor ligula sed lacus. Duis cursus enim ut augue. Cras ac magna. Cras nulla.

Nulla egestas. Curabitur a leo. Quisque egestas wisi eget nunc. Nam feugiat lacus vel est. Curabitur consectetuer.
Suspendisse vel felis. Ut lorem lorem, interdum eu, tincidunt sit amet, laoreet vitae, arcu. Aenean faucibus pede eu
ante. Praesent enim elit, rutrum at, molestie non, nonummy vel, nisl. Ut lectus eros, malesuada sit amet, fermentum
eu, sodales cursus, magna. Donec eu purus. Quisque vehicula, urna sed ultricies auctor, pede lorem egestas dui, et
convallis elit erat sed nulla. Donec luctus. Curabitur et nunc. Aliquam dolor odio, commodo pretium, ultricies non,
pharetra in, velit. Integer arcu est, nonummy in, fermentum faucibus, egestas vel, odio.

\bmsection{Section title of second appendix\label{app2}}%
\vspace*{12pt}
Fusce mauris. Vestibulum luctus nibh at lectus. Sed bibendum, nulla a faucibus semper, leo velit ultricies tellus, ac
venenatis arcu wisi vel nisl. Vestibulum diam. Aliquam pellentesque, augue quis sagittis posuere, turpis lacus congue
quam, in hendrerit risus eros eget felis. Maecenas eget erat in sapien mattis porttitor. Vestibulum porttitor. Nulla
facilisi. Sed a turpis eu lacus commodo facilisis. Morbi fringilla, wisi in dignissim interdum, justo lectus sagittis dui, et
vehicula libero dui cursus dui. Mauris tempor ligula sed lacus. Duis cursus enim ut augue. Cras ac magna. Cras nulla (Figure~\ref{fig5}).

Nulla egestas. Curabitur a leo. Quisque egestas wisi eget nunc. Nam feugiat lacus vel est. Curabitur consectetuer.
Suspendisse vel felis. Ut lorem lorem, interdum eu, tincidunt sit amet, laoreet vitae, arcu. Aenean faucibus pede eu
ante. Praesent enim elit, rutrum at, molestie non, nonummy vel, nisl. Ut lectus eros, malesuada sit amet, fermentum
eu, sodales cursus, magna. Donec eu purus. Quisque vehicula, urna sed ultricies auctor, pede lorem egestas dui, et
convallis elit erat sed nulla. Donec luctus. Curabitur et nunc. Aliquam dolor odio, commodo pretium, ultricies non,
pharetra in, velit. Integer arcu est, nonummy in, fermentum faucibus, egestas vel, odio.

%== Figure 4 ==
%% Example for figure inside appendix
\begin{figure}[b]
\centerline{\includegraphics[height=10pc,width=78mm,draft]{empty}}
\caption{This is an example for appendix figure.\label{fig5}}
\end{figure}

\bmsubsection{Subsection title of second appendix\label{app2.1a}}

Sed commodo posuere pede. Mauris ut est. Ut quis purus. Sed ac odio. Sed vehicula hendrerit sem. Duis non odio.
Morbi ut dui. Sed accumsan risus eget odio. In hac habitasse platea dictumst. Pellentesque non elit. Fusce sed justo
eu urna porta tincidunt. Mauris felis odio, sollicitudin sed, volutpat a, ornare ac, erat. Morbi quis dolor. Donec
pellentesque, erat ac sagittis semper, nunc dui lobortis purus, quis congue purus metus ultricies tellus. Proin et quam.
Class aptent taciti sociosqu ad litora torquent per conubia nostra, per inceptos hymenaeos. Praesent sapien turpis,
fermentum vel, eleifend faucibus, vehicula eu, lacus.

Pellentesque habitant morbi tristique senectus et netus et malesuada fames ac turpis egestas. Donec odio elit,
dictum in, hendrerit sit amet, egestas sed, leo. Praesent feugiat sapien aliquet odio. Integer vitae justo. Aliquam
vestibulum fringilla lorem. Sed neque lectus, consectetuer at, consectetuer sed, eleifend ac, lectus. Nulla facilisi.
Pellentesque eget lectus. Proin eu metus. Sed porttitor. In hac habitasse platea dictumst. Suspendisse eu lectus. Ut
mi mi, lacinia sit amet, placerat et, mollis vitae, dui. Sed ante tellus, tristique ut, iaculis eu, malesuada ac, dui.
Mauris nibh leo, facilisis non, adipiscing quis, ultrices a, dui.

\bmsubsubsection{Subsection title of second appendix\label{app2.1.1a}}

Lorem ipsum dolor sit amet, consectetuer adipiscing elit. Ut purus elit, vestibulum ut, placerat ac, adipiscing vitae,
felis. Curabitur dictum gravida mauris. Nam arcu libero, nonummy eget, consectetuer id, vulputate a, magna. Donec
vehicula augue eu neque. Pellentesque habitant morbi tristique senectus et netus et malesuada fames ac turpis egestas.
Mauris ut leo. Cras viverra metus rhoncus sem. Nulla et lectus vestibulum urna fringilla ultrices. Phasellus eu tellus
sit amet tortor gravida placerat. Integer sapien est, iaculis in, pretium quis, viverra ac, nunc. Praesent eget sem vel
leo ultrices bibendum. Aenean faucibus. Morbi dolor nulla, malesuada eu, pulvinar at, mollis ac, nulla. Curabitur
auctor semper nulla. Donec varius orci eget risus. Duis nibh mi, congue eu, accumsan eleifend, sagittis quis, diam.
Duis eget orci sit amet orci dignissim rutrum (Table~\ref{tab4}).

Nam dui ligula, fringilla a, euismod sodales, sollicitudin vel, wisi. Morbi auctor lorem non justo. Nam lacus libero,
pretium at, lobortis vitae, ultricies et, tellus. Donec aliquet, tortor sed accumsan bibendum, erat ligula aliquet magna,
vitae ornare odio metus a mi. Morbi ac orci et nisl hendrerit mollis. Suspendisse ut massa. Cras nec ante. Pellentesque
a nulla. Cum sociis natoque penatibus et magnis dis parturient montes, nascetur ridiculus mus. Aliquam tincidunt
urna. Nulla ullamcorper vestibulum turpis. Pellentesque cursus luctus mauris.

\begin{table*}[t]%
\centering
\caption{This is an example of Appendix table showing food requirements of army, navy and airforce.\label{tab4}}%
\begin{tabular*}{\textwidth}{@{\extracolsep\fill}llllll@{\extracolsep\fill}}%
\toprule
\textbf{Col1 head} & \textbf{Col2 head} & \textbf{Col3 head} & \textbf{Col4 head} & \textbf{Col5 head} & \textbf{Col6 head} \\
\midrule
col1 text & col2 text & col3 text & col4 text & col5 text & col6 text\\
col1 text & col2 text & col3 text & col4 text & col5 text & col6 text\\
col1 text & col2 text & col3 text& col4 text & col5 text & col6 text\\
\bottomrule
\end{tabular*}
\end{table*}


Example for an equation inside appendix
\begin{equation}
{\mathcal{L}} = i \bar{\psi} \gamma^\mu D_\mu \psi - \frac{1}{4} F_{\mu\nu}^a F^{a\mu\nu} - m \bar{\psi} \psi\label{eq25}
\end{equation}

\bmsection{Example of another appendix section\label{app3}}%
\vspace*{12pt}
This is sample for paragraph text this is sample for paragraph text  this is sample for paragraph text this is sample for paragraph text this is sample for paragraph text this is sample for paragraph text this is sample for paragraph text this is sample for paragraph text this is sample for paragraph text this is sample for paragraph text this is sample for paragraph text this is sample for paragraph text this is sample for paragraph text this is sample for paragraph text this is sample for paragraph text this is sample for paragraph text this is sample for paragraph text this is sample for paragraph text this is sample for paragraph text this is sample for paragraph text this is sample for paragraph text this is sample for paragraph text this is sample for paragraph text this is sample for paragraph text this is sample for paragraph text this is sample for paragraph text this is sample for paragraph text this is sample for paragraph text this is sample for paragraph text this is sample for paragraph text this is sample for paragraph text this is sample for paragraph text this is sample for paragraph text



Nam dui ligula, fringilla a, euismod sodales, sollicitudin vel, wisi. Morbi auctor lorem non justo. Nam lacus libero,
pretium at, lobortis vitae, ultricies et, tellus. Donec aliquet, tortor sed accumsan bibendum, erat ligula aliquet magna,
vitae ornare odio metus a mi. Morbi ac orci et nisl hendrerit mollis. Suspendisse ut massa. Cras nec ante. Pellentesque
a nulla. Cum sociis natoque penatibus et magnis dis parturient montes, nascetur ridiculus mus. Aliquam tincidunt
urna. Nulla ullamcorper vestibulum turpis. Pellentesque cursus luctus mauris.

Nulla malesuada porttitor diam. Donec felis erat, congue non, volutpat at, tincidunt tristique, libero. Vivamus
viverra fermentum felis. Donec nonummy pellentesque ante. Phasellus adipiscing semper elit. Proin fermentum massa
ac quam. Sed diam turpis, molestie vitae, placerat a, molestie nec, leo. Maecenas lacinia. Nam ipsum ligula, eleifend
at, accumsan nec, suscipit a, ipsum. Morbi blandit ligula feugiat magna. Nunc eleifend consequat lorem. Sed lacinia
nulla vitae enim. Pellentesque tincidunt purus vel magna. Integer non enim. Praesent euismod nunc eu purus. Donec
bibendum quam in tellus. Nullam cursus pulvinar lectus. Donec et mi. Nam vulputate metus eu enim. Vestibulum
pellentesque felis eu massa.
\begin{equation}
\mathcal{L} = i \bar{\psi} \gamma^\mu D_\mu \psi
    - \frac{1}{4} F_{\mu\nu}^a F^{a\mu\nu} - m \bar{\psi} \psi
\label{eq26}
\end{equation}

Nulla malesuada porttitor diam. Donec felis erat, congue non, volutpat at, tincidunt tristique, libero. Vivamus
viverra fermentum felis. Donec nonummy pellentesque ante. Phasellus adipiscing semper elit. Proin fermentum massa
ac quam. Sed diam turpis, molestie vitae, placerat a, molestie nec, leo. Maecenas lacinia. Nam ipsum ligula, eleifend
at, accumsan nec, suscipit a, ipsum. Morbi blandit ligula feugiat magna. Nunc eleifend consequat lorem. Sed lacinia
nulla vitae enim. Pellentesque tincidunt purus vel magna. Integer non enim. Praesent euismod nunc eu purus. Donec
bibendum quam in tellus. Nullam cursus pulvinar lectus. Donec et mi. Nam vulputate metus eu enim. Vestibulum
pellentesque felis eu massa.

Quisque ullamcorper placerat ipsum. Cras nibh. Morbi vel justo vitae lacus tincidunt ultrices. Lorem ipsum dolor sit
amet, consectetuer adipiscing elit. In hac habitasse platea dictumst. Integer tempus convallis augue. Etiam facilisis.
Nunc elementum fermentum wisi. Aenean placerat. Ut imperdiet, enim sed gravida sollicitudin, felis odio placerat
quam, ac pulvinar elit purus eget enim. Nunc vitae tortor. Proin tempus nibh sit amet nisl. Vivamus quis tortor
vitae risus porta vehicula.


\begin{center}
\begin{tabular*}{250pt}{@{\extracolsep\fill}lcc@{\extracolsep\fill}}%
\toprule
\textbf{Col1 head} & \textbf{Col2 head} & \textbf{Col3 head} \\
\midrule
col1 text & col2 text & col3 text \\
col1 text & col2 text & col3 text \\
col1 text & col2 text & col3 text \\
\bottomrule
\end{tabular*}
\end{center}


Quisque ullamcorper placerat ipsum. Cras nibh. Morbi vel justo vitae lacus tincidunt ultrices. Lorem ipsum dolor sit
amet, consectetuer adipiscing elit. In hac habitasse platea dictumst. Integer tempus convallis augue. Etiam facilisis.
Nunc elementum fermentum wisi. Aenean placerat. Ut imperdiet, enim sed gravida sollicitudin, felis odio placerat
quam, ac pulvinar elit purus eget enim. Nunc vitae tortor. Proin tempus nibh sit amet nisl. Vivamus quis tortor
vitae risus porta vehicula.

Fusce mauris. Vestibulum luctus nibh at lectus. Sed bibendum, nulla a faucibus semper, leo velit ultricies tellus, ac
venenatis arcu wisi vel nisl. Vestibulum diam. Aliquam pellentesque, augue quis sagittis posuere, turpis lacus congue
quam, in hendrerit risus eros eget felis. Maecenas eget erat in sapien mattis porttitor. Vestibulum porttitor. Nulla
facilisi. Sed a turpis eu lacus commodo facilisis. Morbi fringilla, wisi in dignissim interdum, justo lectus sagittis dui, evehicula libero dui cursus dui. Mauris tempor ligula sed lacus. Duis cursus enim ut augue. Cras ac magna. Cras nulla.
Nulla egestas. Curabitur a leo. Quisque egestas wisi eget nunc. Nam feugiat lacus vel est. Curabitur consectetuer.

Pellentesque habitant morbi tristique senectus et netus et malesuada fames ac turpis egestas. Donec odio elit,
dictum in, hendrerit sit amet, egestas sed, leo. Praesent feugiat sapien aliquet odio. Integer vitae justo. Aliquam
vestibulum fringilla lorem. Sed neque lectus, consectetuer at, consectetuer sed, eleifend ac, lectus. Nulla facilisi.
Pellentesque eget lectus. Proin eu metus. Sed porttitor. In hac habitasse platea dictumst. Suspendisse eu lectus. Ut
mi mi, lacinia sit amet, placerat et, mollis vitae, dui. Sed ante tellus, tristique ut, iaculis eu, malesuada ac, dui.
Mauris nibh leo, facilisis non, adipiscing quis, ultrices a, dui.

\nocite{*}% Show all bib entries - both cited and uncited; comment this line to view only cited bib entries;


\bmsection*{Author Biography}

\begin{biography}{\includegraphics[width=76pt,height=76pt,draft]{empty}}{
{\textbf{Author Name.} Please check with the journal's author guidelines whether
author biographies are required. They are usually only included for
review-type articles, and typically require photos and brief
biographies for each author.}}
\end{biography}


\end{document}
