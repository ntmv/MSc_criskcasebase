 %% The following is a directive for TeXShop to indicate the main file
%%!TEX root = diss.tex

\chapter{Introduction}
\label{ch:Introduction}

\begin{flushleft}
%\begin{epigraph}
    \textbf{Reid:} How do you feel about the cottage industry that’s grown up around it [the Cox model]?
    \\
\textbf{Cox:} Don’t know, really. In the light of some of the further results one knows since, I think I would normally want to tackle problems parametrically, so I would take the underlying hazard to be a Weibull or something. I’m not keen on nonparametric formulations usually.
\\
\textbf{Reid:} So if you had a set of censored survival data today, you might rather fit a parametric model, even though there was a feeling among the medical statisticians that that wasn’t quite right.
\\
\textbf{Cox:} That’s right, but since then various people have shown that the answers are very insensitive to the parametric formulation of the underlying distribution [see, e.g., Cox and Oakes, Analysis of Survival Data, Chapter 8.5]. And if you want to do things like predict the outcome for a particular patient, it’s much more convenient to do that parametrically.

- \textit{Sir. David Cox's reflections on the semi-parametric proportional hazards model (Reid, 1994)}
%\end{epigraph}
\end{flushleft}

In many clinical studies, the main outcome that is being modelled is the time to an event of interest or the \textit{survival} time within the study time period. If the event is experienced by all the individuals in a study, many methods of statistical analyses would be applicable. However, some of the difficulties faced in analyzing survival data are as follows: 1) not all the patients experience the event of interest within the study period (but are still at \textit{risk} to do so in the future). Thus, their true time to the event of interest is unknown. 2) survival data are rarely normally distributed and generally have a skewed shape. For these reasons, the development of methods tailored in particular to \textit{survival data} are necessary. Typically, in the analysis of survival or time-to-event data, we analyze data where the outcome variable denotes the time to occurrence of one event of interest. Often, however, a combined endpoint is a common extension of typical survival analysis, that is, a having cause of interest and a competing risk.  In this context, a competing risk is an event whose occurrence precludes the occurrence of the primary event of interest. For instance, medical studies often consider ‘disease-free survival’, i.e., time until (recurrence of a) disease or death (without prior disease), whichever comes first. In this scenario, while death due to prior disease is our primary cause of interest, death without prior disease (of another cause) would be considered a competing risk.
\medskip \par
From a clinical view-point, the most important quantity of interest from a survival analysis to both clinicians and patients is an estimate of the risk of an event to a patient, given a particular covariate profile. This risk can be obtained by estimating the cumulative incidence function (CIF) (also known as the absolute risk), which is the probability that a person at risk will experience the event of interest within a specified time-interval. One of the most widely used methods that has been developed for the survival analysis of a single-endpoint is the Cox Proportional Hazards model. The flexibility of this model comes from its semi-parametric formulation, which enables the convenient inference of the influence of the covariates on the hazard function, without restraining the response to a particular parametric family. However, this flexibility comes at a cost: the baseline hazard is de-coupled from the effect of the covariates. Thus, the baseline hazard needs to be estimated separately for calculating the cumulative incidence function. This results in step-wise estimates of survival risk, which are often difficult to interpret. Additionally, extending the cox proportional hazards model to the competing risks scenario would involve fitting a cox proportional hazards model for each of the competing risks. This approach is limited by the fact that while the cause-specific hazards can estimated, we cannot estimate the \textit{cumulative incidence} of the event of the interest while simultaneously accounting for competing risks. For this reason, the Fine-Gray model was proposed as an alternative, as this approach directly models the effect of the covariates on the cumulative incidence function. However, when interested in whether a given risk factor or characteristic is associated with the rate of the occurrence of the outcome in subjects who are currently event‐free, we would be interested in the cause-specific hazards as well, which in this model would not be easily available. Furthermore, with certain combinations of covariates, the Fine-Gray model often gives estimates of the cumulative incidence probability that exceed 1 (which we additionally demonstrate here). Indeed, in modelling competing risks, all the available approaches either model the cause-specific hazards or the cumulative incidence function, and while a complete competing risks survival analysis requires both, the process to estimate the other is not quite straight-forward in the currently available models. The generation of survival data, particularly high-dimensional data is becoming more commonplace, and thus, we are in need of a competing risks model that can model the cause-specific hazards, generate an easily interpretable smooth estimate of the cumulative incidence function, and builds a general framework that can be easily extended to applications such as variable selection in high-dimensional data, as well as situations such as time-varying covariates. Thus, in this thesis, we propose to use a multinomial model in combination with \textit{casebase sampling} to fit fully parametric hazard models. This approach comes with the benefits of generating smooth-in-time hazard and cumulative incidence functions. Furthermore, our use of the multinomial regression means that we have access to the nice properties of the generalized linear models family, including variable selection through regularization. In this thesis, we show that the multinomial model with casebase sampling offers good variable selection performance, in both low and high-dimensional data, and additionally shows good performance in predicting the cumulative risk, making this model a viable option for performing a complete competing risks analysis. This thesis is organized as follows: in chapter 2, we give a detailed background of competing risks survival analysis and the currently existing approaches for this kind of data. In chapter 3, we then discuss the mathematical and computational details of the casebase multinomial implementation. In chapter 4, we discuss the results of our simulation experiments, looking at the variable selection, prediction performance, as well as looking into two case-studies, which highlight desirable properties of casebase in relation to other competitors. Finally we conclude this thesis in chapter 5 with mentions of some limitations and extensions of the approach and the simulations performed for future work. 

 
